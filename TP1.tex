\documentclass[a4paper,10pt]{article}
\usepackage[utf8x]{inputenc}
\usepackage{amsmath}
\usepackage{amsfonts}
\usepackage{amssymb}
\usepackage[margin=2cm, top=1.5cm, bottom=1.5cm]{geometry}
% \usepackage{multicol}
\usepackage{graphicx}
\usepackage{calc}
% \usepackage{slashbox}
\usepackage{bbold}
\usepackage{array}
\usepackage{minted}
\usepackage{alltt}
\usepackage{pstricks-add}
\usepackage{pstricks}

% \usepackage{fancyhdr}
% \pagestyle{fancy}
% \fancyhead{}
% \lhead{}
% \rhead{}
% \chead{}
% \renewcommand{\headrulewidth}{0pt}
% \renewcommand{\footrulewidth}{0.4pt}
% \rfoot{\footnotesize{TP 1}}
% \cfoot{}

\title{TP 1 - Analyse des données multi-dimensionnelle}
\author{Guillaume BOULAND\\ Camille MOTTIER}
\date{}

\newcommand{\R}{\mathbb{R}}
\newcommand{\vs}[1]{\vspace{#1cm}}
\newcommand{\hs}[1]{\hspace{#1cm}}
% 
\newcommand{\dsum}[2]{\displaystyle\sum_{#1}^{#2}}
% \newcommand{\dint}[2]{\displaystyle\int_{#1}^{#2}}
\newcommand{\ben}{\begin{enumerate}}
	\newcommand{\een}{\end{enumerate}}
% \newcommand{\bit}{\begin{itemize}}
	% \newcommand{\eit}{\end{itemize}}


\setlength{\parindent}{0pt}


\begin{document}
\maketitle
	
\
	
Toutes les valeurs présentées dans ce document seront données avec une précision de $10^{-3}$.
\section{Partie I}
\subsection{Données quantitatives : étude du nuage de points}
Nous étudions ici des données relatives aux vins de Loire, contenues dans le fichier \texttt{wine.csv}. Dans ce fichier, sont présentés 21 vins selon le type d'appellations, de sol et de 29 paramètres quantitatifs d'intensités sensorielles. 
	
\
	
\begin{table}[ht]
	\centering
	\begin{tabular}{rlllrr}
		\hline
		Label & Soil & Odor.Int.bf.shake & Aroma.qual.bf.shake & Fruity.bf.shake \\ 
		\hline
		2EL  & Saumur & Env1 & 3.074 & 3.000 & 2.714 \\ 
		1CHA & Saumur & Env1 & 2.964 & 2.821 & 2.375 \\ 
		1FON & Bourgueil & Env1 & 2.857 & 2.929 & 2.560 \\ 
		1VAU & Chinon & Env2 & 2.808 & 2.593 & 2.417 \\
		1DAM & Saumur & Reference & 3.607 & 3.429 & 3.154 \\
		\hline
	\end{tabular}
	\caption{Extrait du tableau d'origine} 
\end{table}
	
\
	
Nous commençons par extraire du tableau les variables quantitatives, que nous centrons et réduisons afin de les ramener à une échelle comparable de valeurs. Nous considérons pour cela que les vins sont associés à un même poids : $\frac 1{21}$. 
Nous appellons $X$ ce nouveau tableau et $(x_i^j)$ ses composantes.
	
\
	
\begin{table}[ht]
	\centering
	\begin{tabular}{rrrrr}
		\hline
		Odor.Int.bf.shake & Aroma.qual.bf.shake & Fruity.bf.shake & Flower.bf.shake & Spice.bf.shake \\ 
		\hline
		-0.131 & -0.228 & -0.000 & 1.591 & -0.135 \\ 
		-0.521 & -1.120 & -1.728 & 1.591 & -1.332 \\ 
		-0.900 & -0.582 & -0.785 & -0.692 & 0.366 \\ 
		-1.073 & -2.256 & -1.514 & -1.028 & 0.721 \\ 
		\hline
	\end{tabular}
	\caption{Extrait des valeurs centrées réduites} 
\end{table}
	
\
	
\underline{Barycentre du nuage de points} :\\
	
Le barycentre $\overline{x}$ de ce nuage de points se trouve à l'origine. En effet, les variables étant centrées, nous avons : 
$$\overline{x}=\left(\overline{x^j}\right)_{1\leqslant j\leqslant 29}=\left(\dsum{i=1}{21}\frac 1{21}x_i^j\right)_{1\leqslant j\leqslant 29}=0.$$
	
En notant $W$ la matrice diagonale de $\mathcal M_{21}(\R)$ des poids de chaque vin et $\mathbb 1$ la matrice colonne de $\mathcal  M_{21,1}(\R)$ dont toutes les composantes sont égales à 1, nous obtenons aussi $\overline x$ par le produit matriciel $\overline x'=X'W\mathbb 1$, ce qui est exploité informatiquement pour vérifier le résultat précédent.\\
	
\begin{table}[ht]
	\centering
	\begin{tabular}{rrrr}
		\hline
		Odor.Int.bf.shake & Aroma.qual.bf.shake & Fruity.bf.shake & Flower.bf.shake  \\ 
		\hline
		-0.000 & 0.000 & -0.000 & 0.000  \\ 
		\hline
	\end{tabular}
	\caption{Barycentre du nuage}
\end{table}
	
\vs1
	
\underline{Inertie du nuage de points} :\\
	
Considérant la métrique canonique $I$ sur $\R^{29}$, nous obtenons par le calcul suivant l'inertie du nuage de points : 
$$\begin{array}{ll}
	In_0\{x_i,\frac 1{21}\}_{1\leqslant i\leqslant 21}
	&=\dsum{i=1}{21}\frac 1{21}|\!|x_i|\!|^2_I\\
	&=\dsum{i=1}{21}\frac 1{21}\dsum{j=1}{29}(x_i^j)^2\\
	&=\dsum{j=1}{29}\dsum{i=1}{21}\frac 1{21}(x_i^j)^2\\
	&=\dsum{j=1}{29}V(x^j) \\
	&=\boxed{29} \hs1 \text{car les $x^j$ sont centrées et réduites.}
\end{array}$$
	
\
	
	
	
\subsection{Partition par appellations}
	
Les vins sont séparés en trois appellations : Bourgueil, Chinon et Saumur. \\
Nous noterons $Y=(y_i^k)\in\mathcal M_{21,3}(\R)$ la matrice dont les colonnes sont les indicatrices d'appellations. \\
Les poids globaux des trois appellations sont obtenus par le produit matriciel $Y'W\mathbb 1$ : \\
%En effet, pour tout $k\in\{1,2,3\}$, $\dsum{i=1}{21}y_i^kw_i=\dsum{i/y_i=k}{}w_i=W^k$.\\
%Nous obtenons alors informatiquement les valeurs suivantes. \\
\begin{table}[ht]
	\centering
	\begin{tabular}{rr}
		\hline
		& Poids \\ 
		\hline
		Bourgueil & 0.286 \\ 
		Chinon & 0.190 \\ 
		Saumur & 0.524 \\ 
		\hline
	\end{tabular}
	\caption{Poids par appellation} 
\end{table}
	
\
	
Les barycentres au sein des trois appellations sont obtenus par le calcul matriciel suivant : 
$$\left(\overline x^k\right)_{1\leqslant k\leqslant 3} =(Y'WY)^{-1}(Y'WX)\in \mathcal M_{3,29}(\R).$$ En effet, $Y'WY$ est la matrice diagonale des poids des appellations et $Y'WX=\left(\dsum{i/y_i=k}{}w_ix_i^j\right)_{k,j}$ est la matrice des ``pseudo-moyennes'' par appellation.\\
\begin{table}[ht]
	\centering
	\begin{tabular}{rrrrrr}
		\hline
		& Odor.Int.bf.shake & Aroma.qual.bf.shake & Fruity.bf.shake & Flower.bf.shake & Spice.bf.shake \\ 
		\hline
		Bourgueil & -0.625 & 0.079 & 0.041 & -0.182 & 0.037 \\ 
		Chinon & -0.595 & -0.835 & 0.187 & -0.287 & -0.198 \\ 
		Saumur & 0.558 & 0.260 & -0.091 & 0.204 & 0.052 \\ 
		\hline
	\end{tabular}
	\caption{Barycentres des variables sensorielles par appellation - Extrait} 
\end{table}
	
Ces trois barycentres ont pour norme :\\
\begin{table}[ht]
	\centering
	\begin{tabular}{rr}
		\hline
		& Normes \\ 
		\hline
		Bourgueil & 3.604 \\ 
		Chinon & 5.400 \\ 
		Saumur & 2.120 \\ 
		\hline
	\end{tabular}
	\caption{Normes des barycentres par appellation} 
\end{table}

\

On obtient alors l'inertie inter-appellations : 
$$In_{0}\{\overline x^k,W^k\}_{1\leqslant k\leqslant 3}=\dsum{k=1}3W^k|\!|\overline x^k|\!|_I^2\simeq  \boxed{3,169}$$

puis le coefficient $R^2$ de la partition des vins en appellations : 
$$R^2=\dfrac{In_{0}\{\overline x^k,W^k\}_{1\leqslant k\leqslant 3}}{In_0\{x_i,\frac 1{21}\}_{1\leqslant i\leqslant 21}}\simeq \boxed{0,109}$$

Nous constatons ainsi que la part des appellations dans les disparités sensorielles des vins est d'environ $11\%$.
	
\vs{.5}
	
\subsection{Influence de l'appellation sur les différentes variables sensorielles}
	
Il s'agit ici d'étudier séparément l'influence des appellations sur chaque variable sensorielle. On calcule donc pour chacune d'entre elles le coefficient $R^2$ :
$$(R^2)^j=\dfrac{\dsum{k=1}3W^k(\overline{x^j}^k)^2}{V(x^j)}=\dsum{k=1}3W^k(\overline{x^j}^k)^2$$
	
\begin{table}[ht]
	\centering
	\begin{tabular}{rrrrr}
		\hline
		Odor.Int.bf.shake & Aroma.qual.bf.shake & Fruity.bf.shake & Flower.bf.shake & Spice.bf.shake \\ 
		\hline
		0.342 & 0.170 & 0.011 & 0.047 & 0.009 \\ 
		\hline
	\end{tabular}
	\caption{$R^2$ des variables sensorielles - Extrait} 
\end{table}
	
\
	
Nous constatons que l'appellation est la moins influente sur les variables intitulées ``Spice.before.shaking'', ``Quality.of.odor'', ``Fruity'' et ``Flower'', avec une influence de moins de $1\%$, tandis qu'elle est la plus influente sur les variables intitulées ``Odor.intensity.before.shaking'', ``Odor.intensity'' et ``Phenolic'', avec une influence de plus de $30\%$. \\
	
\begin{table}[ht]
	\centering
	\begin{tabular}{rrrr}
		\hline
		Spice.before.shaking & Quality.of.odour & Fruity & Flower \\ 
		\hline
		0.009 & 0.008 & 0.007 & 0.009 \\ 
		\hline
	\end{tabular}
	\caption{Variables les moins liées à l'appellation} 
\end{table}
	
\
	
\begin{table}[ht]
	\centering
	\begin{tabular}{rrr}
		\hline
		Odor.Int.before.shaking & Odor.Intensity & Phenolic \\ 
		\hline
		0.342 & 0.391 & 0.360 \\ 
		\hline
	\end{tabular}
	\caption{Variables les plus liées à l'appellation} 
\end{table}
	
\
	
Remarquons que le $R^2$ de la partition est égal à la moyenne arithmétique des $R^2$ des variables sensorielles : 
$$\begin{array}{ll}
	R^2
	&=\dfrac{In_{0}\{\overline x^k,W^k\}_{1\leqslant k\leqslant 3}}{In_0\{x_i,\frac 1{21}\}_{1\leqslant i\leqslant 21}}\\
	&= \dfrac 1{29}\dsum{k=1}3W^k\dsum{j=1}{29}(\overline{x^j}^k)^2\\
	&=\dfrac 1{29}\dsum{j=1}{29}\dsum{k=1}3W^k(\overline{x^j}^k)^2\\
	&=\dfrac 1{29}\dsum{j=1}{29}(R^2)^j
\end{array}$$
	
\
	
\begin{table}[!ht]
	\centering
	\begin{tabular}{rr}
		\hline
		moy\_R2var & R2 \\ 
		\hline
		0.109 & 0.109 \\ 
		\hline
	\end{tabular}
	\caption{Vérification de l'égalité entre le $R^2$ et la moyenne arithmétique du $R^2$ des variables} 
\end{table}
	
\vs1
	
\section{Partie II}
	
Dans cette partie, nous étudions l'influence des appellations et de la nature du sol sur les différentes variables sensorielles. Pour ce faire, nous utilisons les projections orthogonales sur les espaces engendrés par les modalités.
	
Nous notons ici $Y$ (respectivement $Z$) la matrice dont les colonnes sont les indicatrices d'appellations (respectivement de sol).
	
\
	
\subsection{Influence de l'appellation }
	
	
\ben
\item 
$\dsum{k=1}3y^k=\mathbb 1$, donc $\mathbb 1\in \,<Y>$. De plus, $<Y^c>\,\subset \mathbb 1^{\bot}$. Donc $<Y>\,=\,<Y^c>\overset{\bot}+<\mathbb 1>$.\\
Or, pour tout $j\in\{1,\ldots,29\}$, $x^j$ est centré donc est orthogonal à $\mathbb 1$. On a ainsi :  
$$\boxed{\Pi_Yx^j=\Pi_{Y^c}x^j+\Pi_{\mathbb 1}x^j=\Pi_{Y^c}x^j}.$$ 
	
De plus, $\Pi_Yx^j=\left(\overline{x^j}^{Y_i}\right)_{1\leqslant i\leqslant 21}$ donc la norme de ce vecteur  peut être interprêtée ainsi : 
$$|\!|\Pi_Yx^j|\!|_W^2=\dsum{k=1}3W^k\left(\overline{x^j}^k\right)^2=\boxed{(R^2)^j}\hs{.5}\text{(car $x^j$ est une variable réduite)}$$
	
	
\item On considère les matrices de $\mathcal M_{21}(\R)$ des projections orthogonales sur $Y$ et sur $x^j$ : 
$$\Pi_Y=Y(Y'WY)^{-1}Y'W ~~
\text{et }\forall j\in\{1,29\},~
\Pi_{x^j} = x^j (\underbrace{{x^j}' Wx^j}_{V(x^j)=1})^{-1} {x^j}' W = x^j {x^j}' W$$
	
%il y avait un conflit entre l'appostrophe et l'exposant j. Avec des accolades, plus de problème...
	
	
La trace du produit de ces deux matrices permet d'obtenir à nouveau les $(R^2)^j$ des variables sensorielles :
$$\text{tr}(\Pi_{x^j}\Pi_Y)=\text{tr}(\Pi_{Y}\Pi_{x^j})=\text{tr}(\Pi_{Y^c}\Pi_{x^j})=[\Pi_{Y^c}|\Pi_{x^j}]=(R^2)^j$$ 
	
	
\begin{table}[ht]
	\centering
	\begin{tabular}{rrrrr}
		\hline
		Odor.Int.bf.shake & Aroma.qual.bf.shake & Fruity.bf.shake & Flower.bf.shake & Spice.bf.shake\\ 
		\hline
		0.342 & 0.170 & 0.011 & 0.047 & 0.009\\ 
		\hline
	\end{tabular}
	\caption{$R^2$ de différentes variables sensorielles - Extrait} 
\end{table}
	
	
\item On introduit la matrice de $\mathcal M_{21}(\R)$ définie par : $R=XMX'W$. On a alors : 
$$\begin{array}{ll}
	\text{tr}(R\Pi_Y)
	&= \text{tr}(XMX'W\Pi_Y)\\\vs{.1}
	&= \dfrac 1{29}\text{tr}(X'W\Pi_YX)\\
	&= \dfrac 1{29}\text{tr}\left(<x^j,\Pi_Yx^k>_W\right)_{j,k}\\
	&= \dfrac 1{29}\dsum{j=1}{29}<x^j,\Pi_Yx^j>_W\\
	&= \dfrac 1{29}\dsum{j=1}{29}|\!|\Pi_Yx^j|\!|_W^2\hs1\text{(car $\Pi_Y$ est auto-adjoint et idempotent})\\\vs{.1}
	&= \dfrac 1{29}(R^2)^j\\
	&=  \boxed{R^2}
\end{array}$$
	
La trace de ce produit matriciel permet de retrouver le coefficient $R^2$ de la partition par appellations.
	
%pour la question d pour les rapprochement avec la partie 1
\begin{table}[ht]
	\centering
	\begin{tabular}{rr}
		\hline
		traceRProjY & R2 \\ 
		\hline
		0.109 & 0.109\\ 
		\hline
	\end{tabular}
	\caption{Vérification de l'égalité $\text{tr}(R\Pi_Y)=R^2$} 
\end{table}
	
\item À l'aide des matrices de projections, et en exploitant les relations obtenues ci-dessus, nous retrouvons informatiquement les résultats de la partie I concernant l'influence des appellations sur les caractéristiques sensorielles des vins.
	
\een
	
\subsection{Influence du sol}
	
En procédant de manière similaire, nous pouvons étudier l'influence du sol sur les caractéristiques sensorielles des vins. Pour ce faire, nous utilisons la  projection sur l'espace engendré par les indicatrices de sols $\Pi_Z$. Nous obtenons alors les $R^2$ pour chaque variable sensorielle mais aussi le $R^2$ global pour les sols : 
	
$$\forall j\in\{1,\ldots,29 \},~(R^2)^j=\text{tr}(\Pi_{x^j}\Pi_Z) ~~\text{ et }~~R^2=\text{tr}(R\Pi_Z)\simeq \boxed{0,365}$$
	
	
\begin{table}[ht]
	\centering
	\begin{tabular}{rrrrr}
		\hline
		Odor.Int.bf.shake & Aroma.qual.bf.shake & Fruity.bf.shake & Flower.bf.shake & Spice.bf.shake \\ 
		\hline
		0.534 & 0.415 & 0.323 & 0.347 & 0.558 \\ 
		\hline
	\end{tabular}
	\caption{$R^2$ de différentes variables sensorielles pour les sols - Extrait} 
\end{table}
	
\
	
\begin{table}[ht]
	\centering
	\begin{tabular}{r}
		\hline
		TraceRProjZ \\ 
		\hline
		0.365 \\ 
		\hline
	\end{tabular}
	\caption{$R^2$ pour le sol} 
\end{table}
	
\
	
	
La nature du sol explique donc environ $37\%$ des disparités sensorielles entre les vins de Loire donc est un élément fortement plus explicatif que l'appellation. 
	
	
	
	
\section{Annexe}
	
\begin{minted}[bgcolor=gray!10]{R}
	wine <- read.csv("wine.csv")
	
	X <- wine[4:32]
	W <- diag(1/21 , nr=21)
	M <- diag(1/29, 29)
	Unit <- matrix(data=c(1), nrow=21, ncol=1)
	
	Xc<- sqrt(21/20)*scale(X)
	X <- Xc
	#Matrice des indicatrices des appellations
	Y <- matrix(0, nrow=21, ncol=3)
	for (i in 1:21) {
		if (wine[i,2]=="Bourgueuil") {Y[i,1]<-1};
		if (wine[i, 2]=="Chinon") {Y[i,2]<-1};
		if (wine[i, 2]=="Saumur") {Y[i,3]<-1}
	}
	#Matrice des indicatrices des sols
	Z <- matrix(0, nrow=21, ncol=4)
	for (i in 1:21) {
		if (wine[i,3]=="Env1") {Z[i,1]<-1};
		if (wine[i, 3]=="Env2") {Z[i,2]<-1};
		if (wine[i, 3]=="Reference") {Z[i,3]<-1};
		if (wine[i, 3]=="Env4") {Z[i, 4]<-1}
	}
	
	#Barycentres des variables sensorielles
	Bar <- t(X) %*% W %*% Unit
	
	#Inertie totale du nuage
	In = sum(apply(1/21*X^2 ,1, sum))
	
	#Poids et barycentre par appellation 
	Poids_app <- t(Y)%*%W%*%Unit
	
	Moy_cat= solve(t(Y)%*%W%*%Y)%*%t(Y)%*%W%*%X
	
	#Calcul normes euclidiennes
	Normes<- apply(Moy_cat^2, 1, sum)
	
	#Calcul de l'inertie inter-appellations
	In_inter <- Normes%*%Poids_app
	
	#Calcul du R^2
	R2 <- In_inter/In
	
	#Variance inter-appellation
	var_inter <- t(Poids_app)%*%Moy_cat^2
	R2var<- var_inter
	
	
	#Variables les moins liées à l'appellation
	subset(R2var, select = R2var<0.01)
	
	#Variables les plus liées à l'appellation
	subset(R2var, select = R2var>0.33)
	
	#Calcul de la moy arithmétique du R2 des variables
	moy_R2var<-mean(R2var)
	
	#Vérifification de l'égalité entre le R2 et la moy arithmétique du R2 des variables
	delta<- moy_R2var - R2
	
	
	#PARTIE 2
	
	#Projection sur Y
	projy <- Y %*% solve(t(Y) %*%W%*% Y) %*% t(Y)%*%W
	
	#fonction de la projection sur les différrentes variables
	projxj<-function(j) {
		proj <- X[,j] %*% solve(t(X[,j]) %*%W%*% X[,j]) %*% t(X[,j])%*%W
		
		return(proj)
	}
	#Matrice de toutes les traces (pour tous les j) du produit Proj_xj*Proj_Y
	V2 <- matrix( 0, nrow=1, ncol=29)
	for (j in 1:29) {
		V2[,j]<- sum(diag(projxj(j)%*%projy))
	}
	
	#Trace du produit R*proj_Y
	R = X%*%M%*%t(X)%*%W
	R_Y = R%*%projy
	traceRY <-sum(diag(R_Y))
	
	#Trace de proj_xj*proj_Z
	projz <- Z %*% solve(t(Z) %*%W%*% Z) %*% t(Z)%*%W
	V3 <- matrix(0, nrow=1, ncol=29)
	for (j in 1:29) {
		V3[,j]<- sum(diag(projxj(j)%*%projz))
	}
	
	#Trace de R*projz
	traceRprojz <- sum(diag(R%*%projz))
\end{minted}
	
	
\end{document}
