\documentclass[a4paper,10pt]{article}
\usepackage[utf8x]{inputenc}
\usepackage{amsmath}
\usepackage{amsfonts}
\usepackage{amssymb}
\usepackage[margin=2cm, bottom=2cm]{geometry}
% \usepackage{multicol}
\usepackage{graphicx}
\usepackage{calc}
% \usepackage{slashbox}
\usepackage{bbold}
\usepackage{array}
      
\usepackage{alltt}
\usepackage{pstricks-add}
\usepackage{pstricks}

% \usepackage{fancyhdr}
% \pagestyle{fancy}
% \fancyhead{}
% \lhead{}
% \rhead{}
% \chead{}
% \renewcommand{\headrulewidth}{0pt}
% \renewcommand{\footrulewidth}{0.4pt}
% \rfoot{\footnotesize{TP 1}}
% \cfoot{}

\title{TP 1 - Analyse des données multi-dimensionnelles}
\author{Guillaume BOULAND\\ Camille MOTTIER}
\date{}

\newcommand{\R}{\mathbb{R}}
\newcommand{\vs}[1]{\vspace{#1cm}}
\newcommand{\hs}[1]{\hspace{#1cm}}
% 
\newcommand{\dsum}[2]{\displaystyle\sum_{#1}^{#2}}
% \newcommand{\dint}[2]{\displaystyle\int_{#1}^{#2}}
% \newcommand{\ben}{\begin{enumerate}}
% \newcommand{\een}{\end{enumerate}}
% \newcommand{\bit}{\begin{itemize}}
% \newcommand{\eit}{\end{itemize}}


\setlength{\parindent}{0pt}


\begin{document}
\maketitle

\

\section{Partie I}
\subsection{Données quantitatives}
Nous étudions ici des données relatives aux vins de Loire, contenues dans le fichier \texttt{wines.csv}. Dans se fichier, sont présentés 21 vins selon le type d'appellations, de sol et de 29 paramètres quantitatifs d'intensités sensorielles. \\

\textit{Extrait du tableau d'origine}\\

Nous commençons par extraire du tableau les variables quantitatives, puis nous les centrons et les réduisons afin de les ramener à une échelle comparable de valeurs. Nous considérons pour cela que les vins sont associés à un même poids : $\frac 1{21}$. 
Nous appellerons par la suite $X$ ce nouveau tableau et $(x_i^j)$ ses composantes.\\

\textit{Extrait du tableau centré-réduit}\\

\underline{Barycentre du nuage de points} :\\

Le barycentre $\overline{x}$ de ce nuage de points se trouve à l'origine. En effet, les variables étant centrées, nous avons : $$\forall j\in[\![1,29]\!],~\overline{x^j}=\dsum{i=1}{21}\frac 1{21}x_i^j=0.$$

En notant $W$ la matrice diagonale de $\mathcal M_{21}(\R)$ des poids de chaque vin et $\mathbb 1$ la matrice colonne de $\mathcal  M_{21,1}(\R)$ composée de 1, nous obtenons aussi $\overline x$ par le produit matriciel $\overline x^T=X^TW\mathbb 1$, ce qui est exploité informatiquement pour vérifier le résultat précédent.\\

\textit{Extrait du résultat obtenu ?} \\

\underline{Inertie du nuage de points} :\\

Considérant la métrique canonique $I$ sur $\R^{29}$, nous obtenons l'inertie du nuage de points : 
$$\begin{array}{ll}
In_0\{x_i,\frac 1{21}\}_{1\leqslant i\leqslant 21}
&=\dsum{i=1}{21}\frac 1{21}|\!|x_i|\!|^2_I\\
&=\dsum{i=1}{21}\frac 1{21}\dsum{j=1}{29}(x_i^j)^2\\
&=\dsum{j=1}{29}\dsum{i=1}{21}\frac 1{21}(x_i^j)^2\\
&=\dsum{j=1}{29}V(x^j) \hs{1} \text{car les $x^j$ sont centrés}\\
&=29 \hs1 \text{car les $x^j$ sont réduits}
\end{array}$$
Résultat que nous retrouvons bien informatiquement. 






\subsection{Partition par appellations}

Les vins sont séparés en trois appellations : Bourgueuil, Chinon et Saumur. \\
Nous noterons $Y=(y_i^k)\in\mathcal M_{21,3}(\R)$ la matrice dont les colonnes sont les indicatrices d'appellations. \\
Les poids des trois appellations sont obtenus par le produit matriciel $Y^TW\mathbb 1$. \\
En effet, pour tout $k\in\{1,2,3\}$, $\dsum{i=1}{21}y_i^kw_i=\dsum{i/y_i=k}{}w_i=W^k$.\\
Nous obtenons alors informatiquement les valeurs suivantes. \\

\textit{Résultat}\\

{\red à développer ???} De manière similaire, les barycentres au sein des trois appellations sont obtenus par le calcul matriciel $(\overline x^k)_{1\leqslant k\leqslant 3} =(Y^TWY)^{-1}(Y^TWX)\in \mathcal M_{3,29}(\R)$. En effet, $Y^TWY$ est la matrice diagonale des poids des appellations et $Y^TWX=\left(\dsum{i/y_i=k}{}w_ix_i^j\right)_{k,j}$ est la matrice des ``pseudo-moyennes'' par appellation.\\

\textit{Extrait des résultats}\\

Ces trois barycentres ont pour norme ...\\

On obtient alors l'inertie inter-appellations : 
$$In_{0}\{\overline x^k,W^k\}_{1\leqslant k\leqslant 3}=\dsum{k=1}3W^k|\!|\overline x^k|\!|_I^2\simeq \text{\red valeur approchée}$$

puis le $R^2$ de la partition des vins en appellations : 
$$R^2=\dfrac{In_{0}\{\overline x^k,W^k\}_{1\leqslant k\leqslant 3}}{In_0\{x_i,\frac 1{21}\}_{1\leqslant i\leqslant 21}}\simeq \text{\red valeur approchée}$$

Nous constatons ainsi que la part des appellations dans les disparités sensorielles des vins est d'environ $11\%$.

\subsection{Influence de l'appellation sur les différentes variables sensorielles}

Il s'agit ici d'étudier séparément l'influence des appellations sur chaque variable sensorielle. On calcule donc pour chacune d'entre elles :
$$(R^2)^j=\dfrac{\dsum{k=1}3W^k(\overline{x^j}^k)^2}{V(x^j)}=\dsum{k=1}3W^k(\overline{x^j}^k)^2$$

\textit{Extrait du résultat}\\

Nous constatons que l'appellation est la moins influente sur les variables intitulées ``Spice.before.shaking'', ``Quality.of.odor'', ``Fruity'' et ``Flower'', avec une influence de moins de $1\%$, tandis qu'elle est la plus influente sur les variables intitulées ``Odor.intensity.before.shaking'', ``Odor.intensity'' et ``Phenolic'', avec une influence de plus de $30\%$. \\

Notons que le $R^2$ de la partition est égal à la moyenne arithmétique des $R^2$ des variables : 
$$\begin{array}{ll}
   R^2
   &=\dfrac{In_{0}\{\overline x^k,W^k\}_{1\leqslant k\leqslant 3}}{In_0\{x_i,\frac 1{21}\}_{1\leqslant i\leqslant 21}}\\
   &= \dfrac 1{29}\dsum{k=1}3W^k\dsum{j=1}{29}(\overline{x^j}^k)^2\\
   &=\dfrac 1{29}\dsum{j=1}{29}\dsum{k=1}3W^k(\overline{x^j}^k)^2\\
   &=\dfrac 1{29}\dsum{j=1}{29}(R^2)^j
  \end{array}$$
ce que nous vérifions aussi informatiquement. 



\section{Partie II}
\section{Annexe}

\end{document}
